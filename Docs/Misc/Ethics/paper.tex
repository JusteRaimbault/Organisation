
\documentclass[10pt]{article}

\usepackage{ragged2e}
\usepackage[margin=2cm]{geometry}


\title{Title}
\author{}
\date{}


\maketitle

\justify


\begin{abstract}

\end{abstract}

% https://www.springer.com/journal/10805 ?


What level of ethics / rigor in self-discipline when it comes to academic ethics? -> examples of minor cases.


\begin{itemize}
	\item Reviewing a paper, literature review is very sparse. Suggests papers in neighbor disciplines. Conflict between disicplines (physics/geography) and maybe between people? Avoid suggesting refs from these people if they are relevant is unethical? if advertising their work made you remove the suggestion, then it should be. However in a broader context, disciplinary positioning, empowering ``weaker'' disciplines, etc.: could it be an argument in favor of this?
	\item What about self-citation ethics, if papers are highly incremental and related?
	\item Reviewing, suggesting your own work may be ok in some cases? (if already cited?)
	\item Self-plagiarism: reuse of work should be avoided as much as possible, but what about the reuse of ideas?
	\item Collaborations: interdisciplinarity and danger of disciplinary arrogance/blindness?
\end{itemize}

Q: which framework could help in all those issues? Ethics of knowledge?












%%%%%%%%%%%%%%%%%%%%
%% Biblio
%%%%%%%%%%%%%%%%%%%%

\bibliographystyle{apalike}
\bibliography{}


\end{document}
