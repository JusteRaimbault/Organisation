% sage_latex_guidelines.tex V1.10, 24 June 2016

\documentclass[Afour,sageh,times]{sagej}

\usepackage{moreverb,url}

\usepackage[colorlinks,bookmarksopen,bookmarksnumbered,citecolor=red,urlcolor=red]{hyperref}

\newcommand\BibTeX{{\rmfamily B\kern-.05em \textsc{i\kern-.025em b}\kern-.08em
T\kern-.1667em\lower.7ex\hbox{E}\kern-.125emX}}

\def\volumeyear{2016}

\begin{document}

\runninghead{Smith and Wittkopf}

\title{A demonstration of the \LaTeXe\ class file for
\itshape{SAGE Publications}}

\author{Alistair Smith\affilnum{1} and Hendrik Wittkopf\affilnum{2}}

\affiliation{\affilnum{1}Sunrise Setting Ltd, UK\\
\affilnum{2}SAGE Publications Ltd, UK}

\corrauth{Alistair Smith, Sunrise Setting Ltd
Brixham Laboratory,
Freshwater Quarry,
Brixham, Devon,
TQ5~8BA, UK.}

\email{alistair.smith@sunrise-setting.co.uk}

\begin{abstract}
This paper describes the use of the \LaTeXe\
\textsf{\journalclass} class file for setting papers to be
submitted to a \textit{SAGE Publications} journal.
The template can be downloaded \href{http://www.uk.sagepub.com/repository/binaries/SAGE LaTeX template.zip}{here}.
\end{abstract}

\keywords{Class file, \LaTeXe, \textit{SAGE Publications}}

\maketitle

\section{Introduction}
Many authors submitting to research journals use \LaTeXe\ to
prepare their papers. This paper describes the
\textsf{\journalclass} class file which can be used to convert
articles produced with other \LaTeXe\ class files into the correct
form for submission to \textit{SAGE Publications}.

The \textsf{\journalclass} class file preserves much of the
standard \LaTeXe\ interface so that any document which was
produced using the standard \LaTeXe\ \textsf{article} style can
easily be converted to work with the \textsf{\journalclassshort}
style. However, the width of text and typesize will vary from that
of \textsf{article.cls}; therefore, \textit{line breaks will change}
and it is likely that displayed mathematics and tabular material
will need re-setting.

In the following sections we describe how to lay out your code to
use \textsf{\journalclass} to reproduce much of the typographical look of
the \textit{SAGE} journal that you wish to submit to. However, this paper is not a guide to
using \LaTeXe\ and we would refer you to any of the many books
available (see, for example, \cite{R1}, \cite{R2} and \cite{R3}).

\section{The three golden rules}
Before we proceed, we would like to stress \textit{three golden
rules} that need to be followed to enable the most efficient use
of your code at the typesetting stage:
\begin{enumerate}
\item[(i)] keep your own macros to an absolute minimum;

\item[(ii)] as \TeX\ is designed to make sensible spacing
decisions by itself, do \textit{not} use explicit horizontal or
vertical spacing commands, except in a few accepted (mostly
mathematical) situations, such as \verb"\," before a
differential~d, or \verb"\quad" to separate an equation from its
qualifier;

\item[(iii)] follow the journal reference style.
\end{enumerate}

\section{Getting started} The \textsf{\journalclassshort} class file should run
on any standard \LaTeXe\ installation. If any of the fonts, style
files or packages it requires are missing from your installation,
they can be found on the \textit{\TeX\ Collection} DVDs or downloaded from
CTAN.

\begin{figure*}
\setlength{\fboxsep}{0pt}%
\setlength{\fboxrule}{0pt}%
\begin{center}
\begin{boxedverbatim}
\documentclass[<options>]{sagej}

\begin{document}

\runninghead{<Author surnames>}

\title{<Initial capital only>}

\author{<An Author\affilnum{1},
Someone Else\affilnum{2} and
Perhaps Another\affilnum{1}>}

\affiliation{<\affilnum{1}First and third authors' affiliation\\
\affilnum{2}Second author affiliation>}

\corrauth{<Corresponding author's name and full postal address>}

\email{<Corresponding author's email address>}

\begin{abstract}
<Text>
\end{abstract}

\keywords{<List keywords>}

\maketitle

\section{Introduction}
.
.
.
\end{boxedverbatim}
\end{center}
\caption{Example header text.\label{F1}}
\end{figure*}

\section{The article header information}
The heading for any file using \textsf{\journalclass} is shown in
Figure~\ref{F1}. You must select options for the trim/text area and
the reference style of the journal you are submitting to.
The choice of \verb+options+ are listed in Table~\ref{T1}.

\begin{table}[h]
\small\sf\centering
\caption{The choice of options.\label{T1}}
\begin{tabular}{lll}
\toprule
Option&Trim and font size&Columns\\
\midrule
\texttt{shortAfour}& 210 $\times$ 280 mm, 10pt& Double column\\
\texttt{Afour} &210 $\times$ 297 mm, 10pt& Double column\\
\texttt{MCfour} &189 $\times$ 246 mm, 10pt& Double column\\
\texttt{PCfour} &170 $\times$ 242 mm, 10pt& Double column\\
\texttt{Royal} &156 $\times$ 234 mm, 10pt& Single column\\
\texttt{Crown} &7.25 $\times$ 9.5 in, 10pt&Single column\\
\texttt{Review} & 156 $\times$ 234 mm, 12pt & Single column\\
\bottomrule
\end{tabular}\\[10pt]
\begin{tabular}{ll}
\toprule
Option&Reference style\\
\midrule
\texttt{sageh}&SAGE Harvard style (author-year)\\
\texttt{sagev}&SAGE Vancouver style (superscript numbers)\\
\texttt{sageapa}&APA style (author-year)\\
\bottomrule
\end{tabular}
\end{table}

For example, if your journal is short A4 sized, uses Times fonts and has Harvard style references then you would need\\
{\small\verb+\documentclass[ShortAfour,times,sageh]{sagej}+}

Most \textit{SAGE} journals are published using Times fonts but if for any reason you have a problem using Times you can
easily resort to Computer Modern fonts by removing the
\verb"times" option.

\subsection{`Review' option}
Some journals (for example, \emph{Journal of the Society for Clinical Trials}) require that 
papers are set single column and with a larger font size to help with the review process. 
If this is a requirement for the journal that you are submitting to, just add the \verb+Review+ option to the \verb+\documenclass[]{sagej}+ line.

\subsection{Remarks}
\begin{enumerate}
\item[(i)] In \verb"\runninghead" use `\textit{et~al.}' if there
are three or more authors.

\item[(ii)] For multiple author papers please note the use of \verb"\affilnum" to
link names and affiliations. The corresponding author details need to be included using the
\verb+\corrauth+ and \verb+\email+ commands.

\item[(iii)] For submitting a double-spaced manuscript, add
\verb"doublespace" as an option to the documentclass line.

\item[(iv)] The abstract should be capable of standing by itself,
in the absence of the body of the article and of the bibliography.
Therefore, it must not contain any reference citations.

\item[(v)] Keywords are separated by commas.

\item[(vi)] If you are submitting to a \textit{SAGE} journal that requires numbered sections (for example, IJRR), please add the command
  \verb+\setcounter{secnumdepth}{3}+ just above the \verb+\begin{document}+ line.

\end{enumerate}


\section{The body of the article}

\subsection{Mathematics} \textsf{\journalclass} makes the full
functionality of \AmS\/\TeX\ available. We encourage the use of
the \verb"align", \verb"gather" and \verb"multline" environments
for displayed mathematics. \textsf{amsthm} is used for setting
theorem-like and proof environments. The usual \verb"\newtheorem"
command needs to be used to set up the environments for your
particular document.

\subsection{Figures and tables} \textsf{\journalclass} includes the
\textsf{graphicx} package for handling figures.

Figures are called in as follows:
\begin{verbatim}
\begin{figure}
\centering
\includegraphics{<figure name>}
\caption{<Figure caption>}
\end{figure}
\end{verbatim}

For further details on how to size figures, etc., with the
\textsf{graphicx} package see, for example, \cite{R1}
or \cite{R3}.

The standard coding for a table is shown in Figure~\ref{F2}.

\begin{figure}
\setlength{\fboxsep}{0pt}%
\setlength{\fboxrule}{0pt}%
\begin{center}
\begin{boxedverbatim}
\begin{table}
\small\sf\centering
\caption{<Table caption.>}
\begin{tabular}{<table alignment>}
\toprule
<column headings>\\
\midrule
<table entries
(separated by & as usual)>\\
<table entries>\\
.
.
.\\
\bottomrule
\end{tabular}
\end{table}
\end{boxedverbatim}
\end{center}
\caption{Example table layout.\label{F2}}
\end{figure}

\subsection{Cross-referencing}
The use of the \LaTeX\ cross-reference system
for figures, tables, equations, etc., is encouraged
(using \verb"\ref{<name>}" and \verb"\label{<name>}").

\subsection{End of paper special sections}
Depending on the requirements of the journal that you are submitting to,
there are macros defined to typeset various special sections. 

The commands available are:
\begin{verbatim}
\begin{acks}
To typeset an
  "Acknowledgements" section.
\end{acks}
\end{verbatim}

\begin{verbatim}
\begin{biog}
To typeset an
  "Author biography" section.
\end{biog}
\end{verbatim}

\begin{verbatim}
\begin{biogs}
To typeset an
  "Author Biographies" section.
\end{biogs}
\end{verbatim}

%\newpage

\begin{verbatim}
\begin{dci}
To typeset a "Declaration of
  conflicting interests" section.
\end{dci}
\end{verbatim}

\begin{verbatim}
\begin{funding}
To typeset a "Funding" section.
\end{funding}
\end{verbatim}

\begin{verbatim}
\begin{sm}
To typeset a
  "Supplemental material" section.
\end{sm}
\end{verbatim}

\subsection{Endnotes}
Most \textit{SAGE} journals use endnotes rather than footnotes, so any notes should be coded as \verb+\endnote{<Text>}+.
Place the command \verb+\theendnotes+ just above the Reference section to typeset the endnotes.

To avoid any confusion for papers that use Vancouver style references,  footnotes/endnotes should be edited into the text.

\subsection{References}
Please note that the files \textsf{SageH.bst} and \textsf{SageV.bst} are included with the class file
for those authors using \BibTeX. 
The files work in a completely standard way, and you just need to uncomment one of the lines in the below example depending on what style you require:
\begin{verbatim}
%%Harvard (name/date)
%\bibliographystyle{SageH}
%%Vancouver (numbered)
%\bibliographystyle{SageV} 
\bibliography{<YourBibfile.bib>} 
\end{verbatim}

%\section{Support for \textsf{\journalclass}}
%We offer on-line support to participating authors. Please contact
%us via e-mail at \dots
%
%We would welcome any feedback, positive or otherwise, on your
%experiences of using \textsf{\journalclass}.

\section{Copyright statement}
Please  be  aware that the use of  this \LaTeXe\ class file is
governed by the following conditions.

\subsection{Copyright}
Copyright \copyright\ \volumeyear\ SAGE Publications Ltd,
1 Oliver's Yard, 55 City Road, London, EC1Y~1SP, UK. All
rights reserved.

\subsection{Rules of use}
This class file is made available for use by authors who wish to
prepare an article for publication in a \textit{SAGE Publications} journal.
The user may not exploit any
part of the class file commercially.

This class file is provided on an \textit{as is}  basis, without
warranties of any kind, either express or implied, including but
not limited to warranties of title, or implied  warranties of
merchantablility or fitness for a particular purpose. There will
be no duty on the author[s] of the software or SAGE Publications Ltd
to correct any errors or defects in the software. Any
statutory  rights you may have remain unaffected by your
acceptance of these rules of use.

\begin{acks}
This class file was developed by Sunrise Setting Ltd,
Brixham, Devon, UK.\\
Website: \url{http://www.sunrise-setting.co.uk}
\end{acks}

\begin{thebibliography}{99}
\bibitem[Kopka and Daly(2003)]{R1}
Kopka~H and Daly~PW (2003) \textit{A Guide to \LaTeX}, 4th~edn.
Addison-Wesley.

\bibitem[Lamport(1994)]{R2}
Lamport~L (1994) \textit{\LaTeX: a Document Preparation System},
2nd~edn. Addison-Wesley.

\bibitem[Mittelbach and Goossens(2004)]{R3}
Mittelbach~F and Goossens~M (2004) \textit{The \LaTeX\ Companion},
2nd~edn. Addison-Wesley.

\end{thebibliography}

\end{document}
